\documentclass{beamer}
\mode<presentation>
{ \usetheme{boxes} }

\usepackage{times}
\usepackage{graphicx}
\usepackage{csquotes}
\usepackage[backend=bibtex]{biblatex}

\title{Dogen v1.0.13, \enquote{Clube Náutico}}

\author{Marco Craveiro \\
  Domain Driven Development
}
\date{\today}

\AtBeginSection[]
{
  \begin{frame}<beamer>
    \frametitle{Outline}
    \tableofcontents[currentsection]
  \end{frame}
}

\bibliography{sprint_13_features}
\begin{document}

\begin{frame}
\titlepage
\end{frame}

\begin{frame}
\frametitle{Tests facet}

\begin{itemize}

  \pause

\item Created mainly for internal purposes, but since it is a facet,
  it is exposed to end users.

  \pause

\item You can enable it on your own models if you want to test dogen
  types, but most likely you'll want it disabled. However, it may be
  useful to report problems.

\end{itemize}

\end{frame}

\begin{frame}
\frametitle{Delete empty directories}

\begin{itemize}

  \pause

\item Enabling and disabling facets results in the creation of lots of
  empty directories. These are not picked up by git.

  \pause

\item New setting now allows the clean up of empty directories.

\end{itemize}

\end{frame}

\begin{frame}
\frametitle{Next Sprint}

\begin{itemize}
\item \textbf{Mission statement}: Move decoration related elements
  into metamodel. Start work on moving annotation profiles into
  metamodel.
\end{itemize}

\end{frame}

\end{document}
