\documentclass{beamer}
\mode<presentation>
{ \usetheme{boxes} }

\usepackage{times}
\usepackage{acronym}
\usepackage{graphicx}
\usepackage[backend=bibtex]{biblatex}

\newcommand{\etal}{\textit{et al}. }
\newcommand{\ie}{\textit{i}.\textit{e}., }
\newcommand{\eg}{\textit{e}.\textit{g}. }
\newcommand{\etc}{\textit{etc}. }

\title{Variability in Transformations}
\author{Marco Craveiro}
\date{\today}

\AtBeginSection[]
{
  \begin{frame}<beamer>
    \frametitle{Outline}
    \tableofcontents[currentsection]
  \end{frame}
}

\bibliography{variability_in_transformations}

\begin{document}

\begin{frame}
\titlepage
v${DOGEN_VERSION}
\end{frame}

\section{Background}

\begin{frame}
\frametitle{What is \ac{MDE}}

\begin{itemize}
\item Previous presentation provided an overview of what \ac{MDE} is
  and why you should consider it for Software Engineering.

\pause

\item \ac{MDE} is a ``development paradigm that uses models as the
  primary artifact of the development process. Usually [\ldots] the
  implementation is (semi-)automatically generated from the
  models.''\cite{brambilla2012model}

\pause

\item \ac{MDE} ``goes beyond the pure development activities and
  encompasses other model-based tasks of a complete software
  engineering process (\eg the model-based evolution of the system or
  the model-driven reverse engineering of a legacy
  system.)''\cite{brambilla2012model}

\end{itemize}

\end{frame}

\begin{frame}
\frametitle{\ac{MDE} Key Components}

\begin{itemize}

\item ``Fundamental'' \ac{MDE} equation: Models + Transformations =
  Software

\pause

\item \ac{MDE} is only interested in a special class of models: those
  which are described according to a \textbf{formal language} and thus
  have precise semantics.

\pause

\item Formal models are written using a \acf{DSL}. The \ac{DSL} is
  designed specifically for modeling. \acf{UML} is an example of such
  a \ac{DSL}. \ac{UML} has a graphical representation~--- called a
  \emph{concrete syntax}, but also an \emph{abstract syntax}.

\end{itemize}

\end{frame}

\begin{frame}
\frametitle{\ac{MDE} Key Components}

\begin{itemize}

\item A \emph{meta-model} is made up of the abstract syntax and the
  static semantics of the \ac{DSL}. The metamodel is the basis for the
  transformations.

\pause

\item \emph{Transformations}~--- or just transforms~--- can be thought
  of as functions that take models as inputs and produce an output.

\pause

\item The output of a transform can be another model; it can either
  \emph{conform} to the same metamodel or to a different metamodel. In
  this case we classify the transform as \ac{M2M}.

\pause

\item The output of the transform can also be text; in this case we
  classify the transform as \ac{M2T}. These are commonly referred to
  as \emph{generators}.

\end{itemize}

\end{frame}

\section{Variability}

\begin{frame}
\frametitle{Software Product Lines}

\begin{itemize}

\item \ac{MDE} ``pursues the goal of creating a software
  \emph{product} in part or in the whole through one or more
  transformations.''\cite{volter2013model}

\pause

\item Once you have a product, the next logical think is to think of
  \emph{product families}: ``the set of all products that can be
  created with a certain domain architecture is commonly referred to
  as a \emph{software system family}.''''\cite{volter2013model}

\pause

\item ``Software product line engineering aims to reduce development
  time, effort, cost, and complexity by taking advantage of the
  commonality within a portfolio of similar
  products.''\cite{voelter2007handling}

\end{itemize}

\end{frame}

\begin{frame}
\frametitle{Software Product Lines}

\begin{itemize}

\item \ac{PLE} integrates with \ac{MDE} seamlessly:

\pause


\end{itemize}

\end{frame}

\begin{frame}
\frametitle{Acronyms}

\begin{acronym}
  \acro{DSL}{Domain Specific Language}
  \acro{GPML}{General Purpose Modeling Languages}
  \acro{M2M}{Model-to-Model}
  \acro{M2T}{Model-to-Text}
  \acro{MDE}{Model-driven Engineering}
  \acro{PIM}{Platform Independent Model}
  \acro{PLE}{Product Line Engineering}
  \acro{PSM}{Platform Specific Model}
  \acro{T2M}{Text-to-Model}
  \acro{UML}{Unified Modeling Language}
\end{acronym}

\end{frame}

\begin{frame}
\frametitle{Bibliography}

\printbibliography

\end{frame}



\end{document}
