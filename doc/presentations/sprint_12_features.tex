\documentclass{beamer}
\mode<presentation>
{ \usetheme{boxes} }

\usepackage{times}
\usepackage{graphicx}
\usepackage{csquotes}
\usepackage[backend=bibtex]{biblatex}

\title{Dogen v1.0.12, \enquote{Estádio do Atlético}}

\author{Marco Craveiro \\
  Domain Driven Development
}
\date{\today}

\AtBeginSection[]
{
  \begin{frame}<beamer>
    \frametitle{Outline}
    \tableofcontents[currentsection]
  \end{frame}
}

\bibliography{sprint_12_features}
\begin{document}

\begin{frame}
\titlepage
\end{frame}

\begin{frame}
\frametitle{Dry-run mode}

\begin{itemize}

\item Useful when you'd like to know what would be changed by running
  code generation against the current state of the model.

  \pause

\item Dogen can report a unified diff, to file or console. The unified
  diff can be used to create a patch for review or for application.

\end{itemize}

\end{frame}

\begin{frame}
\frametitle{Transitive References}

\begin{itemize}

\item In the past you had to manually add all references to the
  top-level model. This was a lazy way to get things working.

  \pause

\item With v1.0.12, Dogen correctly resolves transitive references, so
  you no longer need to do this.

  \pause

\item In the future, we'll add warnings to let you know when a
  reference is not required.

\end{itemize}

\end{frame}

\begin{frame}
\frametitle{Reporting}

\begin{itemize}

\item It is sometimes useful to know what all the operations are that
  Dogen has carried out or, in the case of dry run, will carry out.

  \pause

\item This is particularly useful for Dogen development, but may also
  be useful to end users.

\end{itemize}

\end{frame}

\begin{frame}
\frametitle{Byproducts directory}

\begin{itemize}

\item Files were scattered across the filesystem, making it difficult
  to find them.

  \pause

\item All files are now placed in a top-level byproducts directory,
  making it easy to both find and get rid of all non-code related
  files.

  \pause

\item Directory is organised by model so that you can have multiple
  runs of different models.

\end{itemize}

\end{frame}

\begin{frame}
\frametitle{Next Sprint}

\begin{itemize}
\item \textbf{Mission statement}: Move all of the decoration related
  code into the meta-model. This means that much of what currently
  exists as assorted files that dogen loads on startup would become
  regular model entities, paving the way for a much more configurable
  model.
\end{itemize}

\end{frame}

\end{document}
